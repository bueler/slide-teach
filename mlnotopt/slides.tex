% Copyright 2024  Ed Bueler

\documentclass[10pt,
               svgnames,
               hyperref={colorlinks,citecolor=DeepPink4,linkcolor=FireBrick,urlcolor=Maroon},
               usepdftitle=false]{beamer}

\mode<presentation>
{
  \usetheme{Madrid}

  \usecolortheme{beaver}

  \setbeamercovered{transparent}
  
  \setbeamerfont{frametitle}{size=\large}
}

\setbeamercolor*{block title}{bg=red!10}
\setbeamercolor*{block body}{bg=red!5}

\usepackage[english]{babel}
\usepackage[latin1]{inputenc}
\usepackage{times}
\usepackage[T1]{fontenc}
% Or whatever. Note that the encoding and the font should match. If T1
% does not look nice, try deleting the line with the fontenc.

\usepackage{empheq,bm,xspace,minted}
\usepackage{hyperref}
\usepackage{tikz}

% If you wish to uncover everything in a step-wise fashion, uncomment
% the following command: 
%\beamerdefaultoverlayspecification{<+->}

\newcommand{\bb}{\mathbf{b}}
\newcommand{\bc}{\mathbf{c}}
\newcommand{\bbf}{\mathbf{f}}
\newcommand{\bl}{\bm{\ell}}
\newcommand{\br}{\mathbf{r}}
\newcommand{\bs}{\mathbf{s}}
\newcommand{\bx}{\mathbf{x}}
\newcommand{\by}{\mathbf{y}}
\newcommand{\bv}{\mathbf{v}}
\newcommand{\bu}{\mathbf{u}}
\newcommand{\bw}{\mathbf{w}}

\newcommand{\bzero}{\bm{0}}

\newcommand{\CC}{\mathbb{C}}
\newcommand{\RR}{\mathbb{R}}

\newcommand{\ddt}[1]{\ensuremath{\frac{\partial #1}{\partial t}}}
\newcommand{\ddx}[1]{\ensuremath{\frac{\partial #1}{\partial x}}}
\renewcommand{\t}[1]{\texttt{#1}}
\newcommand{\Matlab}{\textsc{Matlab}\xspace}
\newcommand{\Octave}{\textsc{Octave}\xspace}
\newcommand{\eps}{\epsilon}

\newcommand{\twovect}[4]{\ensuremath{{#1}_{#2} =
                            \begin{bmatrix} #3 \\ #4 \end{bmatrix}}}

\newcommand{\ftt}[1]{{\color{blue} \texttt{#1}}}

\newcommand{\rbullet}{{\color{FireBrick} \bullet}}

\newcommand*\circled[1]{\tikz[baseline=(char.base)]{
            \node[shape=circle,draw,inner sep=2pt] (char) {#1};}}


\title{Training in machine learning is not optimization}

\subtitle{but it is close}

\author{Ed Bueler}

\institute[]{UAF Math 661 Optimization}

\date{Fall 2024}


\begin{document}
\beamertemplatenavigationsymbolsempty

\begin{frame}
  \maketitle
\end{frame}

\begin{frame}{Outline}
  \tableofcontents[hideallsubsections]
\end{frame}

\section{how fast is the basic matrix-vector product $z=Ax$?}


\begin{frame}[fragile]
\frametitle{fast solution of circulant systems}

\begin{itemize}
\item suppose $A$ is circulant, with first column $c$, and we want to solve $Ax=b$
\item we know
  $$A = c_0 I + c_1 D_m + c_2 D_m^2 + \dots + c_{m-1} D_m^{m-1} = p(D_m)$$
and $D_m = F_m \Lambda F_m^{-1}$ where $\lambda_j = {\bar\omega_m}^j$
\item fast solution process:
    $$Ax = b \qquad \iff \qquad F_m p(\Lambda) F_m^{-1} x = b \qquad \iff \qquad \begin{matrix} u = F_m^{-1} b \\ v = p(\Lambda)^{-1} u \\ x = F_m v \end{matrix}$$ 

\begin{minted}[fontsize=\small]{python}
def solve_circulant(c,x,b):
    z = barFFT(c)
    u = barFFT(b)
    v = u ./ z
    x = FFT(v)
    x /= m
    return x
\end{minted}

\vspace{-20mm}
\hfill $\gets$ \quad $O(m\log m)$ flops \dots optimal!

\vspace{15mm}
\end{itemize}
\end{frame}


\begin{frame}{summary: some directions suggested by this talk}

\begin{itemize}
\item sparse storage schemes
   \begin{itemize}
   \item[$\circ$] practicalities?
   \item[$\circ$] parallelization?
   \end{itemize}
\item sparse direct linear algebra
   \begin{itemize}
   \item[$\circ$] how to re-order variables to minimize fill-in in LU
   \item[$\circ$] this is graph theory (nested-disection, minimum degree, Cuthill-McKee, \dots)
   \end{itemize}
\item how does the FFT actually work?
   \begin{itemize}
   \item[$\circ$] why is it $O(m\log m)$?
   \item[$\circ$] what else is it good for? (signal/image processing, filters, fast Poisson, \dots)
   \end{itemize}
\item Krylov methods
   \begin{itemize}
   \item[$\circ$] conjugate gradient, GMRES, \dots
   \item[$\circ$] matrix-free Krylov
   \end{itemize}
\item matrix-based preconditioners for Krylov methods
   \begin{itemize}
   \item[$\circ$] incomplete LU, incomplete Cholesky, \dots
   \end{itemize}
\item multigrid for PDE problems
   \begin{itemize}
   \item[$\circ$] geometric multigrid
   \item[$\circ$] algebraic multigrid, a black-box-ish preconditioner
   \end{itemize}
\end{itemize}
\end{frame}


\begin{frame}{references}

\begin{itemize}
{\small
%\item[] \textbf{E.~Bueler (2021)}. \emph{PETSc for Partial Differential Equations}, SIAM Press, Philadelphia
%    \begin{itemize}
%    \item[$\circ$] Krylov methods, multigrid, optimal PDE solvers
%    \end{itemize}
\item[] \textbf{L.~Trefethen \& D.~Bau (2022)}. \emph{Numerical Linear Algebra}, 25th anniversary ed., SIAM Press, Philadelphia
    \begin{itemize}
    \item[$\circ$] clear thinking on matrices and core algorithms
    \end{itemize}
}
\end{itemize}

\end{frame}

\end{document}
