\documentclass[12pt]{amsart}
%prepared in AMSLaTeX, under LaTeX2e
\addtolength{\oddsidemargin}{-.5in}
\addtolength{\evensidemargin}{-.5in}
\addtolength{\topmargin}{-.6in}
\addtolength{\textwidth}{1.0in}
\addtolength{\textheight}{0.9in}
\newcommand{\normalspacing}{\renewcommand{\baselinestretch}{1.1}
        \tiny\normalsize}

\newtheorem*{thm}{Theorem}
\newtheorem*{defn}{Definition}
\newtheorem*{example}{Example}
\newtheorem*{problem}{Problem}
\newtheorem*{remark}{Remark}

\usepackage{amssymb,fancyvrb,alltt,xspace}

\usepackage{palatino}

\usepackage[final]{graphicx}

\usepackage[pdftex, colorlinks=true, plainpages=false, linkcolor=black, citecolor=red, urlcolor=red]{hyperref}

% macros
\newcommand{\CC}{\mathbb{C}}
\newcommand{\Div}{\nabla\cdot}
\newcommand{\eps}{\epsilon}
\newcommand{\grad}{\nabla}
\newcommand{\ZZ}{\mathbb{Z}}
\newcommand{\ip}[2]{\ensuremath{\left<#1,#2\right>}}
\newcommand{\lam}{\lambda}
\newcommand{\lap}{\triangle}
\newcommand{\RR}{\mathbb{R}}

\newcommand{\prob}[1]{\bigskip\noindent\large\textbf{#1}.\,\normalsize }
\newcommand{\ppart}[1]{\textbf{(#1)}\,\, }
\newcommand{\epart}[1]{\medskip\noindent\textbf{(#1)}\,\, }

\newcommand{\Matlab}{\textsc{Matlab}\xspace}
\newcommand{\Octave}{\textsc{Octave}\xspace}
\newcommand{\MO}{\Matlab/\Octave}


\begin{document}
\scriptsize \noindent Math 426 Numerical Analysis (Bueler) \hfill \today
\normalsize\bigskip

\Large\centerline{\textbf{Assignment \#3}}
\normalsize\medskip

\large
\centerline{\textbf{Due Friday 20 September at start of class.}}
\normalsize\bigskip
\thispagestyle{empty}

\renewcommand{\baselinestretch}{0.5}  \tiny\normalsize
\begin{quote}\small
These exercises are \textbf{not} based on the course textbook.  Instead, you should read the online slides at
  \begin{center}
  \href{https://bueler.github.io/numerical/assets/slides/F24/polynonewt.pdf}{\texttt{bueler.github.io/numerical/assets/slides/F24/polynonewt.pdf}}
  \end{center}
\normalsize
\end{quote}
\normalspacing

\bigskip

\prob{1}  Consider 3 points:  $\{(1,1),\, (2.5,8), \,(4,5)\}$.  Find the polynomial $P(x)$ of lowest degree which passes through these points.  Do this two different ways, by using

\epart{a} the Vandermonde matrix method (implemented \emph{without} \texttt{polyfit})

\epart{b} the Lagrange form.

\medskip\noindent\emph{Comment}.  You may use \Matlab to do the computations, including solving linear systems via the backslash operation, or by hand.  Please make sure to show me the matrix entries in part \textbf{(a)}.  Also show me the Lagrange polynomials in part \textbf{(b)} before you combine them to give $P(x)$.

\epart{c}  Now show, by hand, that the polynomials in parts \textbf{(a)} and \textbf{(b)} are identical.  For instance, you can put both answers in standard (monomial) form, which should be already done in part \textbf{(a)}.


\prob{2}  Write a short \Matlab program which generates 6 random points and then uses the \texttt{vander} command to construct the polynomial $P(x)$ through these points.  Generate the points this way:
\begin{Verbatim}[frame=single,fontfamily=courier,fontsize=\scriptsize]
  x = randn(1,6)
  y = randn(1,6)
\end{Verbatim}
Your program should print out the coefficients in some clear manner.  Then, in a figure similar to the one on the ``did we solve the problem?'' slide, plot both the polynomial and the points it goes through.

\medskip
\noindent \emph{Practical comment}.  Once you have generated a figure you can put it into a printable PDF by the command \quad \texttt{>> print -dpdf myfigure.pdf}


\prob{3}  Consider the function $f(x) = 2^x$ on the interval $[0.5,3]$.

\epart{a} Find the polynomial $P(x)$ of degree 4 which interpolates $f(x)$ at the 5 equally-spaced points $1,1.5,2,2.5,3$.  (\emph{Use either method, Vandermonde or Lagrange.})

\epart{b} Find the polynomial $Q(x)$ of degree 4 which interpolates $f(x)$ at the 5 \emph{un}equally-spaced points $1,1.2,1.5,2.8,3$.  (\emph{Again, use the method of your choice.})

\epart{c} Make a plot which shows $f(x)$, $P(x)$, and $Q(x)$ on the same axes.  By eyeball, and between $P(x)$ and $Q(x)$, which is more accurate as an approximation of $f(x)$?
\end{document}