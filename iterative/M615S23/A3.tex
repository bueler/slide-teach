\documentclass[12pt]{amsart}
%prepared in AMSLaTeX, under LaTeX2e
\addtolength{\oddsidemargin}{-.6in} 
\addtolength{\evensidemargin}{-.6in}
\addtolength{\topmargin}{-.4in}
\addtolength{\textwidth}{1.2in}
\addtolength{\textheight}{.6in}

\renewcommand{\baselinestretch}{1.05}

\usepackage{verbatim,fancyvrb,bm}

\usepackage{palatino}

\newtheorem*{thm}{Theorem}
\newtheorem*{defn}{Definition}
\newtheorem*{example}{Example}
\newtheorem*{problem}{Problem}
\newtheorem*{remark}{Remark}

\usepackage{xspace}
\usepackage{amssymb}

\usepackage[pdftex, colorlinks=true, plainpages=false, linkcolor=blue, citecolor=red, urlcolor=blue]{hyperref}

% macros

\newcommand{\bb}{\mathbf{b}}
\newcommand{\bbf}{\mathbf{f}}
\newcommand{\br}{\mathbf{r}}
\newcommand{\bs}{\mathbf{s}}
\newcommand{\bv}{\mathbf{v}}
\newcommand{\bx}{\mathbf{x}}
\newcommand{\by}{\mathbf{y}}

\newcommand{\bzero}{\bm{0}}

\newcommand{\CC}{\mathbb{C}}
\newcommand{\RR}{\mathbb{R}}
\newcommand{\ZZ}{\mathbb{Z}}

\newcommand{\eps}{\epsilon}
\newcommand{\grad}{\nabla}
\newcommand{\lam}{\lambda}
\newcommand{\lap}{\triangle}

\newcommand{\ip}[2]{\ensuremath{\left<#1,#2\right>}}

%\renewcommand{\det}{\operatorname{det}}
\newcommand{\onull}{\operatorname{null}}
\newcommand{\rank}{\operatorname{rank}}
\newcommand{\range}{\operatorname{range}}

\newcommand{\prob}[1]{\bigskip\noindent\textbf{#1.}\quad }
\newcommand{\exer}[2]{\prob{Exercise #2 in Lecture #1}}

\newcommand{\pts}[1]{(\emph{#1 pts}) }
\newcommand{\epart}[1]{\medskip\noindent\textbf{(#1)}\quad }
\newcommand{\ppart}[1]{\,\textbf{(#1)}\quad }

\newcommand{\Matlab}{\textsc{Matlab}\xspace}

\DefineVerbatimEnvironment{mVerb}{Verbatim}{numbersep=2mm,
frame=lines,framerule=0.1mm,framesep=2mm,xleftmargin=4mm,fontsize=\footnotesize}


\begin{document}
\scriptsize \noindent Math 615 NADE (Bueler) \hfill 8 February, 2023
\normalsize

\medskip\bigskip

\Large\centerline{\textbf{Assignment \#3}}
\large
\bigskip

\centerline{\textbf{Due Wednesday, 15 February 2023, at the start of class}}
\bigskip
\normalsize

\thispagestyle{empty}

\bigskip
There will be no lectures on 8 \& 10 February (Wednesday \& Friday).  Instead, please carefully go through the following slides:

\medskip
\centerline{\href{http://bueler.github.io/nade/assets/slides/iterative.pdf}{\texttt{bueler.github.io/nade/assets/slides/iterative.pdf}}}

\medskip
\noindent  This Assignment is based on these slides, and previous material.  However, sections 4.1, 4.2, and 2.16 in the textbook complement this material.

\prob{P12}  \ppart{a} Write a \Matlab/etc.~function for Richardson iteration, with signature

\bigskip
\centerline{\texttt{function z = richardson(A,b,x0,N,omega)}}

\bigskip
\noindent It should return the $N$th iterate $\bx_N$ as \texttt{z}.  Confirm that it works by showing you get the same $\bx_3$ as on page 4 of the slides.

\epart{b}  How many iterations are needed to get 8 digit accuracy for LS1 with $\bx_0=0$ and using the preferred value of $\omega$?  (\emph{Clearly state how you interpret ``8 digit accuracy''.})  How many iterations for $\omega = 0.1$ and $\omega = 0.5$?


\prob{P13}  \ppart{a}  Write \Matlab functions which do $N$ iterations of the Jacobi and Gauss-Seidel (GS) methods:

\medskip
\centerline{\texttt{function z = jacobi(A,b,x0,N)}}

\centerline{\texttt{function z = gs(A,b,x0,N)}}

\medskip
\noindent For each one use the entries of $A$ directly.  That is, for \texttt{jacobi()}, implement formula (5) from the slides, and for \texttt{gs()} implement formula (7).  (\emph{Do not split $A=D-L-U$ and store those parts; this is a waste of memory and misses the point.})  Your implementation of GS should use less memory than Jacobi; make sure this is clear. 

\epart{b}  For each method, how many iterations are needed to get 8 digit accuracy for LS1 using $\bx_0=0$?  (\emph{Clearly state how you interpret ``8 digit accuracy''.})

\epart{c}  Demonstrate that GS fails on LS2.  Now compute an explanatory spectral radius.


%\prob{PX}  For $N$ iterations on an $m\times m$ matrix $A$, how many floating-point operations (additions, subtractions, multiplications, divisions) does Jacobi iteration require?  Use formula (5) to answer.  Next answer the same question for GS, using formula (7).  (\emph{Your answers will be in terms of $N$ and $m$.})

\prob{P14}  Show that Jacobi iteration converges if $A$ is strictly diagonally-dominant.  (\emph{Hints:  Jacobi iteration converges if and only if $\rho(M) < 1$ for $M = - D^{-1}(L+U)$.  So suppose $M\bv = \lambda \bv$ for $\bv\ne 0$.  Choose the largest-magnitude entry $v_i$ of $\bv$, so that $|v_i| \ge |v_j|$ for all $j$.  Show then that $M\bv=\lambda\bv$, and the assumption of strict diagonal dominance, shows $|\lambda v_i| < |v_i|$ which shows $|\lambda|<1$.})


\prob{P15}  \ppart{a}  Consider this boundary value problem from \textbf{P10} on Assignment \#2:
\begin{equation*}
u''(x) + q\, u(x) = f(x), \quad u(x_L) = \alpha, \quad u(x_R) = \beta. \end{equation*}
Implement the centered finite difference method for this problem.  Your code should have the signature

\centerline{\texttt{function [x,u] = bvpq(m,xL,xR,q,f,alpha,beta)}}

\noindent where input \texttt{f} is the function $f(x)$, but the other inputs are integers or real numbers.  The outputs are the grid vector \texttt{x} and the (approximate) solution vector \texttt{u}.  In this initial implementation, your code should use \Matlab's backslash command, or similar built-in solver, to solve the linear system.  (\emph{Include the codes from parts \textbf{(a)} and \textbf{(d)} in what you turn in.})

\epart{b}  Check correctness of \texttt{bvpq} by solving the problem
\begin{equation*}
u''(x) - u(x) = f(x), \quad u(0) = 1, \quad u(2) = 0.
\end{equation*}
exactly, using the solution $u_{\text{ex}}(x)=1 - \sin(\pi x/4)$.  That is, start by finding the $f(x)$ for which this $u_{\text{ex}}(x)$ is the exact solution.  (\emph{This is using the method of} manufactured solutions.)  Show a figure which confirms that you have a verified code.

\epart{c}  Your part \textbf{(a)} code sets up and solves a linear system $AU=F$.  For what $q$ values is $A$ strictly diagonally-dominant (SDD)?

\epart{d}  Duplicate the code from part \textbf{(a)}, give it a new name \texttt{bvpqgs}, and implement Gauss-Seidel (GS) to solve the linear system, instead of calling the built-in linear solver.  Let $x_L,x_R,f(x),\alpha,\beta$ all be as in part \textbf{(b)}.  For each of $m=5$ and $m=50$ find nonzero values $q$ where Gauss-Seidel does converge and does not converge.  (\emph{That is, find 4 values of $q$ with these properties.})  When convergence happens, how many iterations give 8 digit accuracy?


\prob{P16}  In calculus you probably learned Newton's method as a memorized formula: $x_{k+1} = x_k - f(x_k)/f'(x_k)$.  Rewrite equations Newton's method equations (8), (9) from the slides, in the one-dimensional case ($n=1$), to derive this memorized formula.


\prob{P17}  \ppart{a}  Consider these 3 equations, chosen for visualizability:
  $$x^2+y^2+z^2 = 4, \quad x = \cos(\pi y), \quad z = y^2$$
Sketch each equation individually as a surface in $\RR^3$.  (\emph{Do this by hand or by computer, the goal being a clear mental image of intersecting surfaces.  Accuracy is not important.})  Describe informally why there are two solutions of this system of three equations, that is, two points $(x,y,z)\in\RR^3$ at which all three equations are satisfied.  Explain why both solutions are inside the box $-1\le x \le 1, -2\le y \le 2, 0\le z \le 2$.

\epart{b}  The slides describe Newton's method for nonlinear systems.  Implement it in \Matlab/etc.~to solve the above nonlinear system.  Show your script and generate at least five iterations.  Use $\bx_0=(-1,1,1)$ as an initial iterate to find one solution, and also find the other solution using a different initial iterate.  Note that \, \texttt{format long} \, is appropriate here.

\end{document}
