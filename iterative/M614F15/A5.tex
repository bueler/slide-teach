\documentclass[12pt]{amsart}
%prepared in AMSLaTeX, under LaTeX2e
\addtolength{\oddsidemargin}{-.65in} 
\addtolength{\evensidemargin}{-.65in}
\addtolength{\topmargin}{-.4in}
\addtolength{\textwidth}{1.3in}
\addtolength{\textheight}{.75in}

\renewcommand{\baselinestretch}{1.05}

\usepackage{verbatim,fancyvrb}

\usepackage{palatino}

\newtheorem*{thm}{Theorem}
\newtheorem*{defn}{Definition}
\newtheorem*{example}{Example}
\newtheorem*{problem}{Problem}
\newtheorem*{remark}{Remark}

\newcommand{\mtt}{\texttt}
\usepackage{alltt,xspace}
\newcommand{\mfile}[1]
{\medskip\begin{quote}\scriptsize \begin{alltt}\input{#1.m}\end{alltt} \normalsize\end{quote}\medskip}

\usepackage[final]{graphicx}
\newcommand{\mfigure}[1]{\includegraphics[height=2.5in,
width=3.5in]{#1.eps}}
\newcommand{\regfigure}[2]{\includegraphics[height=#2in,
keepaspectratio=true]{#1.eps}}
\newcommand{\widefigure}[3]{\includegraphics[height=#2in,
width=#3in]{#1.eps}}

\usepackage{amssymb}

\usepackage[pdftex, colorlinks=true, plainpages=false, linkcolor=blue, citecolor=red, urlcolor=blue]{hyperref}

% macros

\newcommand{\bb}{\mathbf{b}}
\newcommand{\br}{\mathbf{r}}
\newcommand{\bv}{\mathbf{v}}
\newcommand{\bx}{\mathbf{x}}
\newcommand{\by}{\mathbf{y}}

\newcommand{\CC}{\mathbb{C}}
\newcommand{\RR}{\mathbb{R}}
\newcommand{\ZZ}{\mathbb{Z}}

\newcommand{\eps}{\epsilon}
\newcommand{\grad}{\nabla}
\newcommand{\lam}{\lambda}
\newcommand{\lap}{\triangle}

\newcommand{\ip}[2]{\ensuremath{\left<#1,#2\right>}}

%\renewcommand{\det}{\operatorname{det}}
\newcommand{\onull}{\operatorname{null}}
\newcommand{\rank}{\operatorname{rank}}
\newcommand{\range}{\operatorname{range}}

\newcommand{\prob}[1]{\bigskip\noindent\textbf{#1.}\quad }
\newcommand{\exer}[2]{\prob{Exercise #2 in Lecture #1}}

\newcommand{\pts}[1]{(\emph{#1 pts}) }
\newcommand{\epart}[1]{\medskip\noindent\textbf{(#1)}\quad }
\newcommand{\ppart}[1]{\,\textbf{(#1)}\quad }

\newcommand{\Matlab}{\textsc{Matlab}\xspace}

\DefineVerbatimEnvironment{mVerb}{Verbatim}{numbersep=2mm,
frame=lines,framerule=0.1mm,framesep=2mm,xleftmargin=4mm,fontsize=\footnotesize}


\begin{document}
\scriptsize \noindent Math 614 Numerical Linear Algebra (Bueler) \hfill 5 October, 2015
\normalsize

\bigskip

\Large\centerline{\textbf{Assignment \#5}}
\large
\medskip

\centerline{\textbf{Due Wednesday 14 October, 2015 at the start of class}}
\bigskip
\normalsize

\thispagestyle{empty}

\bigskip
There will be no class on Friday 9 October, 2015.  Please read slides at

\medskip
\centerline{\href{http://bueler.github.io/M614F15/iterative.pdf}{\texttt{bueler.github.io/M614F15/iterative.pdf}}}

\medskip

\prob{P10}  Use \Matlab to compute the $2$-norm condition numbers for systems LS1 and LS2 in the slides.  (\emph{Thereby confirm that these systems have unique solutions which can be well-approximated.})  Find the exact solutions of these systems.  (\emph{For example, use \Matlab any way you want, and then check that solution by-hand.})

\prob{P11}  Write a \Matlab function for Richardson iteration, with first line

\bigskip
\centerline{\texttt{function z = richardson(A,b,x0,omega,N)}}

\bigskip
\noindent It should return the $N$th iterate $\bx_N$ as \texttt{z}.  Confirm that it works by showing you get the same $\bx_3$ as on page 4 of the slides.  What is a preferred value for $\omega$ in system LS1?  How many iterations are needed to get 8 digit accuracy for LS1 with $\bx_0=0$ and this preferred value of $\omega$?
% 22 iterations using omega=0.4 with norm(z-zexact)  [2-norm]

\prob{P12}  Find a small example matrix $A$ which has all zeros on the diagonal but which is invertible.  Find its inverse.

\prob{P13}  Write \emph{two} \Matlab functions for Gauss-Seidel iteration with first lines

\bigskip
\centerline{\texttt{function z = gs1(A,b,x0,N)}}

\centerline{\texttt{function z = gs2(A,b,x0,N)}}

\bigskip
\noindent For \texttt{gs1()}, implement formula (7) from the slides by carefully using \Matlab functions \texttt{triu()} and \texttt{tril()} to extract the parts of $A$, and then using backslash.  For \texttt{gs2()} implement (8) by using only scalar arithmetic operations, and \texttt{for} loops.  (I.e.~pretend it is old Fortran.)

Demonstrate that the two versions work identically on LS1 by computing two iterations with each.  How many iterations are needed to get 8 digit accuracy for LS1 using $\bx_0=0$?
% 16 iterations with norm(z-zexact)
After demonstrating that Gauss-Seidel iteration fails on LS2, compute a spectral radius that explains why it fails.
%>> max(abs(eig(tril(A2)\triu(A2,1))))
%ans =     6.85410196624969             ... is >> 1

\prob{P14}  Show that Jacobi iteration converges if $A$ is strictly diagonally-dominant.  (\emph{Hints:  Jacobi iteration converges if and only if $\rho(M) < 1$ for $M = - D^{-1}(L+U)$.  So suppose $M\bv = \lambda \bv$ for $\bv\ne 0$.  Choose the largest-magnitude entry $v_i$ of $\bv$, so that $|v_i| \ge |v_j|$ for all $j$.  Show then that $M\bv=\lambda\bv$, and the assumption of strict diagonal dominance, shows $|\lambda v_i| < |v_i|$ which shows $|\lambda|<1$.})

\end{document}
