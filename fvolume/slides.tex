% Copyright 2020  Ed Bueler

\documentclass[10pt,hyperref]{beamer}

\mode<presentation>{
  \usetheme{Madrid}
  \usecolortheme{beaver}
  \setbeamercovered{transparent}  
  \setbeamerfont{frametitle}{size=\large}
}

\setbeamercolor*{block title}{bg=red!10}
\setbeamercolor*{block body}{bg=red!5}

\usepackage[english]{babel}
\usepackage[latin1]{inputenc}
\usepackage{times}
\usepackage[T1]{fontenc}
% Or whatever. Note that the encoding and the font should match. If T1
% does not look nice, try deleting the line with the fontenc.

\usepackage{empheq}
\usepackage{xspace}
\usepackage{verbatim,fancyvrb}

\usepackage{tikz}
\usetikzlibrary{shapes,arrows.meta,decorations.markings,decorations.pathreplacing,fadings,positioning}

\usepackage{hyperref}

% If you wish to uncover everything in a step-wise fashion, uncomment
% the following command: 
%\beamerdefaultoverlayspecification{<+->}

\newcommand{\ba}{\mathbf{a}}
\newcommand{\bb}{\mathbf{b}}
\newcommand{\bc}{\mathbf{c}}
\newcommand{\bg}{\mathbf{g}}
\newcommand{\bq}{\mathbf{q}}
\newcommand{\br}{\mathbf{r}}
\newcommand{\bx}{\mathbf{x}}
\newcommand{\by}{\mathbf{y}}
\newcommand{\bv}{\mathbf{v}}
\newcommand{\bu}{\mathbf{u}}
\newcommand{\bw}{\mathbf{w}}

\newcommand{\bF}{\mathbf{F}}

\newcommand{\grad}{\nabla}
\newcommand{\Div}{\nabla\cdot}

\newcommand{\CC}{\mathbb{C}}
\newcommand{\RR}{\mathbb{R}}

\newcommand{\ddt}[1]{\ensuremath{\frac{\partial #1}{\partial t}}}
\newcommand{\ddx}[1]{\ensuremath{\frac{\partial #1}{\partial x}}}
\newcommand{\Matlab}{\textsc{Matlab}\xspace}
\newcommand{\Octave}{\textsc{Octave}\xspace}
\newcommand{\eps}{\epsilon}

\newcommand{\ip}[2]{\left<#1,#2\right>}

\newcommand{\trefcolumn}[1]{\begin{bmatrix} \phantom{x} \\ #1 \\ \phantom{x} \end{bmatrix}}
\newcommand{\trefmatrixtwo}[2]{\left[\begin{array}{c|c|c} & & \\ #1 & \dots & #2 \\ & & \end{array}\right]}
\newcommand{\trefmatrixthree}[3]{\left[\begin{array}{c|c|c|c} & & & \\ #1 & #2 & \dots & #3 \\ & & & \end{array}\right]}
\newcommand{\trefmatrixgroups}[4]{\left[\begin{array}{c|c|c|c|c|c} & & & & & \\ #1 & \dots & #2 & #3 & \dots & #4 \\ & & & & & \end{array}\right]}

\newcommand{\blocktwo}[4]{\left[\begin{array}{c|c} #1 & #2 \\ \hline #3 & #4 \end{array}\right]}

\newcommand{\bqed}{{\color{blue}\qed}}
\newcommand{\ds}{\displaystyle}

\newcommand\mynum[1]{{\renewcommand{\insertenumlabel}{#1}%
      \usebeamertemplate{enumerate item} \,}}


\AtBeginSection[]
{
  \begin{frame}<beamer>
    \frametitle{Outline}
    \tableofcontents[currentsection,hideallsubsections]
  \end{frame}
}

\title[Finite volume methods]{Finite volume methods for \\ advection equations and hyperbolic systems}

%\subtitle{part I: xx}

\author{Ed Bueler}

\institute[UAF]{University of Alaska Fairbanks}

\date{June 2020}


\begin{document}
\beamertemplatenavigationsymbolsempty

\begin{frame}
  \maketitle
\end{frame}

\begin{frame}
  \frametitle{Outline}
  \tableofcontents[hideallsubsections]
\end{frame}


\begin{frame}[fragile]
\frametitle{overview}

\begin{itemize}
\item numerical solutions of first-order PDEs
    \begin{itemize}
    \item[$\circ$] finite volume (FV) methods
    \item[$\circ$] a genuine introduction!
    \end{itemize}
\item we will solve these hyperbolic equations:

\bigskip
\begin{center}
\begin{tikzpicture}[scale=0.9,
                    >={Latex[length=2mm]},
  eqn/.style={
     rectangle,draw,fill=white,align=center,line width=0.6pt,minimum width=15mm}]

%center
\draw[line width=1pt] (0,0)     node[eqn] (scalaradvect)  {\mynum{1} scalar advection \\ {\Large \strut} $u_t + a u_x=0$};
\draw[line width=1pt] (3,-1.7)     node[eqn] (scalartwod)  {\mynum{4} scalar advection in 2D \\ {\Large \strut} $u_t + a u_x + b u_y=0$};
\draw[line width=1pt] (-2,-1.7)    node[eqn] (linearsystem)  {\mynum{2} linear system \\ {\Large \strut} $\bq_t + A\, \bq_x=0$};
\draw[line width=1pt] (-2,-3.55)    node[eqn] (generalsystem)  {\mynum{5} conservation-law system \\ {\Large \strut} $\bq_t + \bF(t,x,\bq)_x=\bg(t,x,\bq)$};

\path[-Latex]
   ([xshift=-2em]scalaradvect.south) edge node {} (linearsystem)
   ([xshift=2em]scalaradvect.south) edge node {} (scalartwod)
   (linearsystem.south) edge node {} (generalsystem);
\end{tikzpicture}

\bigskip
    \begin{itemize}
    \item[$\circ$] ``hyperbolic'' means finite speed of influence
    \item[$\circ$] arrows show generalizations
    \end{itemize}
\end{center}
\end{itemize}
\end{frame}


\begin{frame}{references}
\begin{itemize}
\item {\large \alert{R.~J.~LeVeque, \emph{Finite Volume Methods for Hyperbolic Problems}, Cambridge University Press, 2002}}

\bigskip
\item K.~W.~Morton and D.~F.~Mayers, \emph{Numerical Solutions of Partial Differential Equations: An Introduction}, Cambridge University Press, 2nd ed., 2005
\item W.~Hundsdorfer and J.~G.~Verwer, \emph{Numerical Solution of Time-Dependent Advection-Diffusion-Reaction Equations}, Springer, 2003
\item E.~Bueler, \emph{PETSc for Partial Differential Equations}, SIAM Press, 2020?
\end{itemize}
\end{frame}


\begin{frame}{visual example 1: numerical solution}

\begin{itemize}
\item advection equation for scalar density $q(t,x)$:
    $$q_t + a q_x = 0$$
with speed $a=1$, initial condition $q(0,x)$ known, and periodic boundary conditions on $0\le x \le 1$
\item movie of numerical solution for $0\le t \le 1$
    \begin{itemize}
    \item[$\circ$] initial shape is transported from initial position back to same position
    \end{itemize}
\end{itemize}

\vspace{10mm}
\begin{center}
\alert{SHOW MOVIE}
\end{center}
\vspace{10mm}

%  ./riemann -problem advection -da_grid_x 200 -limiter minmod -ts_monitor_solution binary:q.dat -ts_monitor binary:t.dat
% mkdir foo
% ./plotTS.py -mx 200 -ylabel "q" -ax 0.0 -bx 1.0 -cellcentered t.dat q.dat -oroot foo/bar
% cd foo
% eog bar*.png
\end{frame}


\begin{frame}{visual example 1: numerical \emph{and exact} solutions}

\begin{itemize}
\item compare:
\end{itemize}

\bigskip
\hfill \mbox{\includegraphics[width=0.48\textwidth]{figs/leveque6p1upwind} \, \includegraphics[width=0.48\textwidth]{figs/leveque6p1lw}}
\end{frame}


\begin{frame}{visual example 2: merely a numerical solution}

\begin{itemize}
\item shallow water water equations:

\vspace{-11.5mm}
\begin{align*}
\hspace{60mm} h_t + (hu)_x &= 0 \\
(hu)_t + \left(h u^2 + \frac{1}{2} g h^2\right)_x &= 0
\end{align*}
\item initial condition is a ``hump'' ($h(0,x)=a e^{-bx^2}$, $u(0,x)=0$) on $x\in[-5,5]$
    \begin{itemize}
    \item[$\circ$] a vertical displacement in the center of the domain
    \item[$\circ$] simplest model for a tsunami from block fault in middle of ocean?
    \end{itemize}
\item movie of numerical solution for $0 \le t \le 18$

\vspace{10mm}
\begin{center}
\alert{SHOW MOVIE}
\end{center}

\vspace{10mm}

% ./riemann -problem swater -initial hump -da_grid_x 1000 -limiter minmod -ts_monitor binary:t.dat -ts_monitor_solution binary:q.dat
% mkdir baz
% ./plotTS.py -mx 1000 -ylabel "h (height)" -ax -5.0 -bx 5.0 -dof 2 -c 0 -cellcentered t.dat q.dat -oroot baz/bar
% cd baz
% eog bar*.png

\item we need exact solutions when designing high-quality simulations
    \begin{itemize}
    \item[$\circ$] precise tools for engineering!
    \item[$\circ$] \emph{verification}
    \end{itemize}
\end{itemize}
\end{frame}


\begin{frame}{high performance?}

\begin{itemize}
\item I am interested in high performance and parallel solutions of PDEs
    \begin{itemize}
    \item[$\circ$] so I am not always using \Matlab or Python/scipy
    \end{itemize}
\item these movies were generated by C programs which call PETSc, the Portable Extensible Toolkit for Scientific computing, a mathematical library for high-performance computing brought to you by the same people who brought you MPI:

    \begin{center}
    \href{https://www.mcs.anl.gov/petsc/}{\texttt{www.mcs.anl.gov/petsc/}}
    \end{center}
\item speed and parallelizability are nice for one spatial dimension, but become critical for 2D and 3D
\end{itemize}

\vspace{10mm}
\begin{center}
\alert{SHOW CREATION OF SHALLOW WATER MOVIE}
\end{center}

\end{frame}


\begin{frame}{please ask questions}

\begin{itemize}
\item the rest of the talk is about the math not the movies
    \begin{itemize}
    \item[$\circ$] but lots of figures to explain concepts
    \end{itemize}
\item \alert{PLEASE} ask lots of questions, about any topic at all
    \begin{itemize}
    \item[$\circ$] slowing me down is a \emph{good} thing!
    \end{itemize}
\end{itemize}
\end{frame}


\section{scalar advection equation}

\begin{frame}{one-way advection}

\begin{itemize}
\item the \emph{scalar advection equation (PDE)} for $q(t,x) \in \RR$:
    $$q_t + a q_x=0$$

    \begin{itemize}
    \item[$\circ$] $a\in\RR$ constant on this slide
    \end{itemize}
\item if we have initial condition $q(t,0)=f(x)$, with $f(x)$ smooth, then the solution is
    $$q(t,x) = f(x-at)$$
\item \emph{because}, by the chain rule
    $$q_t(t,x) = f'(x-at) (-a) \quad \text{ and } \quad q_x(t,x) = f'(x-at)$$
so $q_t + a q_x = -a f'(x-at) + a f'(x-at) = 0$
\item the solution $q(t,x)=f(x-at)$ is a movie of the graph of $f(x)$ translating to the right by $at$ in time $t$
    \begin{itemize}
    \item[$\circ$] even if $a$ and/or $t$ are negative
    \item[$\circ$] note $a$ is the speed of the motion
    \end{itemize}
\end{itemize}
\end{frame}


\begin{frame}{solution by characteristics}

\begin{itemize}
\item but what about the scalar PDE
    $$q_t + a(x) q_x=0$$
where $a(x)$ is a spatially-variable speed
\item now we need to expose the idea of a \emph{characteristic curve}:
    $$\frac{dx}{dt} = a(x(t))$$

    \begin{itemize}
    \item[$\circ$] this is an ODE for $x(t) \in \RR$
    \end{itemize}
\item FIXME solve by characteristics $\frac{d}{dt} q(t,x(t)) = q_t(t,x(t)) + q_x(t,x(t)) x'(t) = q_t(t,x(t)) + q_x(t,x(t)) a(x(t)) = 0$
\item FIXME show solution of $q_t + a(t,x) q_x = g(t,x)$
\end{itemize}
\end{frame}


\begin{frame}{upwind scheme}

\begin{itemize}
\item apply \emph{upwind} scheme to $q_t + aq_x=0$ in case $a>0$:
    $$\frac{Q_j^{n+1} - Q_j^n}{\Delta t} + a \frac{Q_j^n - Q_{j-1}^n}{\Delta x} = 0$$
\item equivalently, solve for new value:
\begin{align*}
Q_j^{n+1} &= \frac{a\Delta x}{\Delta t} Q_{j-1}^n + \left(1 - \frac{a\Delta x}{\Delta t}\right) Q_j^n = \ell(x_j-a\Delta t)
\end{align*}
where $\ell(x)$ linearly interpolates

\noindent between $(x_{j-1},Q_{j-1}^n)$ and $(x_{j},Q_{j}^n)$

\vspace{-9mm}
\hfill \includegraphics[width=0.45\textwidth]{figs/leveque4p4a}

\vspace{-7mm}
\item interpolate $\ell$ at the characteristic

\noindent through $(x_{j},Q_j^{n+1})$
\item except we must require \emph{interpolation} instead of extrapolation: $\ds \frac{|a|\Delta x}{\Delta t} \le 1$
\end{itemize}
\end{frame}


\begin{frame}{upwind and Lax-Wendroff schemes}

\begin{itemize}
\item Lax-Wendroff FIXME is quadratic interpolation
\end{itemize}
\end{frame}


\begin{frame}{results}

\begin{itemize}
\item x

\hfill \includegraphics[width=0.7\textwidth]{figs/leveque6p1}
\end{itemize}
\end{frame}


\begin{frame}{the finite volume idea}

\begin{itemize}
\item x
\end{itemize}

\begin{center}
\includegraphics[width=0.75\textwidth]{figs/leveque4p1}
\end{center}
\end{frame}


\begin{frame}{method-of-lines thinking (MOL)}

\begin{itemize}
\item x
\end{itemize}
\end{frame}


\begin{frame}{x}

\begin{itemize}
\item x
\end{itemize}
\end{frame}


\begin{frame}{x}

\begin{itemize}
\item x
\end{itemize}
\end{frame}



\section{linear systems}

\begin{frame}{linear systems}

\begin{itemize}
\item system for $\bq(t,x) \in \RR^n$ with $A\in\RR^{n\times n}$ constant:
  $$\bq_t + A\, \bq_x=0$$
\item for example,
    \begin{itemize}
    \item[$\circ$] \emph{acoustics} (and classic wave equation)
        $$\bq = \begin{bmatrix} p \\ u \end{bmatrix}, \,\, A = \begin{bmatrix} 0 & K_0 \\ \frac{1}{\rho_0} & 0 \end{bmatrix} \quad \implies \quad \begin{matrix} p_t + K_0 u_x = 0 \\ u_t + \frac{1}{\rho_0} p_x = 0 \end{matrix} \qquad \begin{pmatrix} \, {\large \strut} p_{tt} = \frac{K_0}{\rho_0} p_{xx}\, \\ \, {\large \strut} u_{tt} = \frac{K_0}{\rho_0} u_{xx}\, \end{pmatrix}$$
    \item[$\circ$] \underline{linearized} \emph{shallow water equations}
        $$\bq = \begin{bmatrix} h \\ h u \end{bmatrix}, \,\, A = \begin{bmatrix} 0 & 1 \\ -u_0^2+gh_0 & 2 u_0 \end{bmatrix} \, \implies \, {\footnotesize \begin{matrix} {\large \strut} h_t + (hu)_x = 0 \\ {\large \strut} (h u)_t + (-u_0^2+gh_0) h_x + 2u_0 (h u)_x = 0 \end{matrix} }$$
    \item[$\circ$] boringly decoupled
        $$\bq = \begin{bmatrix} u \\ v \\ w \end{bmatrix}, \,\, A = \begin{bmatrix} a & 0 & 0 \\ 0 & b & 0 \\ 0 & 0 & c \end{bmatrix} \, \implies \begin{matrix} u_t + a u_x = 0 \\ v_t + b v_x = 0 \\ w_t + c w_x = 0 \end{matrix}$$
    \end{itemize}
\end{itemize}
\end{frame}



\begin{frame}{example: acoustics}

\begin{itemize}
\item $p(t,x)$ is gas pressure, $u(t,x)$ is gas velocity
\item constant density $\rho_0$ and modulus of compressiblity $K_0$
\item assume pressure/velocity variations are small
\item linear, constant-coefficient first-order PDE system:
$$\begin{matrix} p_t + K_0 u_x = 0 \\ u_t + \frac{1}{\rho_0} p_x = 0 \end{matrix} \hspace{60mm}$$
or $\bq_t + A \bq_x$ where
$$\bq = \begin{bmatrix} p \\ u \end{bmatrix}, \, A = \begin{bmatrix} 0 & K_0 \\ \frac{1}{\rho_0} & 0 \end{bmatrix} \hspace{65mm}$$
\item example solution $\longrightarrow$

\vspace{-30mm}
\hfill \includegraphics[width=0.55\textwidth]{figs/leveque3p1}
\end{itemize}
\end{frame}


\begin{frame}{hyperbolic systems}

\begin{itemize}
\item the system $\bq_t + A\, \bq_x=0$ is \emph{hyperbolic} if $A$ is diagonalizable and all of the eigenvalues of $A$ are real
    \begin{itemize}
    \item[$\circ$] diagonalizable means there are eigenvectors of $A$ which form a basis of $\RR^n$
    \end{itemize}
\item consider \emph{left eigenvectors} of $A$, namely vectors $\bw_k \in \RR^n$ so that
    $$\bw_k^\top A = \lambda_k \bw_k^\top$$

\vspace{-2mm}
    \begin{itemize}
    \item[$\circ$] $\lambda_k$ are \emph{eigenvalues}, real numbers if system is hyperbolic
    \item[$\circ$] $\bw$ are column vectors so $\bw_k^\top$ are row vectors
    \item[$\circ$] $\bw$ are the right eigenvectors of $A^\top$: \qquad $A^\top \bw_k = \lambda_k \bq_k$
    \end{itemize}
\end{itemize}
\end{frame}


\begin{frame}{eigenvectors decouple systems}

\begin{itemize}
\item decouple the system $\bq_t + A\, \bq_x=0$ by multiplying by $\bw_k^\top$:
\begin{align*}
\bw_k^\top \bq_t + \bw_k^\top A\, \bq_x &= 0 \\
\bw_k^\top \bq_t + \lambda_k \bw_k^\top \bq_x &= 0 \\
(v_k)_t + \lambda_k (v_k)_x &= 0
\end{align*}
\item so the scalar functions $v_k(t,x) = \bw_k^\top \bq(t,x)$ satisfy decoupled advection eqns:
   $$(v_k)_t + \lambda_k (v_k)_x = 0$$
\end{itemize}
\end{frame}


\begin{frame}{inner products, left eigenvectors, transposes, and \Matlab}

\begin{itemize}
\item $\bu^\top \bv = \ip{\bu}{\bv}$
\item left eigenvectors for $A$ are the same as right eigenvectors for $A^\top$
\item basis of left iff basis of right
\item in \Matlab, find left eigenvectors this way
\end{itemize}
\end{frame}


\begin{frame}{Riemann solver}

\begin{itemize}
\item x
\end{itemize}
\end{frame}


\begin{frame}{two ways of thinking}

\begin{itemize}
\item Riemann solver at cell face $(t,x_{j+1/2})$ versus characteristic to $(t_{n+1},x_j)$
\end{itemize}
\end{frame}


\begin{frame}{Riemann solver for the acoustics problem}

\begin{itemize}
\item FIXME
\end{itemize}
\end{frame}


\section{high-resolution methods}

\begin{frame}{Godunov's barrier theorem}

\begin{itemize}
\item can we do better for the simple problem $q_t + a q_x=0$?
\item can we have higher accuracy without oscillations?
    \begin{itemize}
    \item[$\circ$] upwinding is only first-order accurate,\footnote{$O(\Delta t + \Delta x)$ local truncation error as $\Delta t\to 0$ and $\Delta x \to 0$} but it avoids oscillations
    \item[$\circ$] Lax-Wendroff is second-order,\footnote{$O(\Delta t^2 + \Delta x^2)$ local truncation error} but it generates oscillations beyond the range of the initial condition
    \end{itemize}
\item \alert{NO.}

\begin{theorem}[\emph{Godunov's barrier theorem, 1959}]  A monotonicity-preserving \alert<2>{linear} scheme for the 1D constant-coefficient equation $q_t + a q_x=0$ cannot have second-order (or higher) local truncation error in $x$.\end{theorem}

\item this theorem is the beginning of modern hyperbolic PDE solvers
    \begin{itemize}
    \item[$\circ$] upwinding, Lax-Friedrichs, Lax-Wendroff, leapfrog are all older
    \item[$\circ$] ``high-resolution'' schemes of the 1970s--1990s finally overcame it
    \item[$\circ$] \alert{how?}
    \end{itemize}
\end{itemize}
\end{frame}


\begin{frame}{slope limiter idea}

\begin{itemize}
\item Gudonov view of upwind
\item overshoot in Lax-Wendroff

\hfill \includegraphics[width=0.75\textwidth]{figs/leveque6p4}
\item actual slope limiter
\end{itemize}
\end{frame}


\begin{frame}{advection equation}

\begin{itemize}
\item consider linear advection again: $q_t + a q_x=0$ with $a$ constant
\end{itemize}

\mbox{\includegraphics[width=0.5\textwidth]{figs/leveque6p1} \, \includegraphics[width=0.5\textwidth]{figs/leveque6p2}}
\end{frame}


\begin{frame}{MOL slope limiting}

\begin{itemize}
\item x
\end{itemize}

\hfill \includegraphics[width=0.75\textwidth]{figs/leveque10p2}
\end{frame}


\section{scalar advection in 2D}

\begin{frame}{x}

\begin{itemize}
\item see \texttt{c/ch11/advect.c} at \href{https://github.com/bueler/p4pdes}{\texttt{github.com/bueler/p4pdes}}
\item x
\end{itemize}
\end{frame}


\section{nonlinear conservation-law systems}

\begin{frame}{traffic}

\begin{itemize}
\item $q_t + F(q)_x = 0$ where $F(q) = U(q) q$ and $U(q) = u_{max} (1-q)$
\end{itemize}

\hfill \includegraphics[width=0.65\textwidth]{figs/leveque11p1}
\end{frame}

\begin{frame}{Riemann solver for traffic equation}

\begin{itemize}
\item $F(q) = U(q) q$ and $U(q) = u_{max} (1-q)$
\end{itemize}

\hfill \includegraphics[width=0.75\textwidth]{figs/leveque12p1}
\end{frame}


\begin{frame}{shallow water equations}

\begin{itemize}
\item $h(t,x)$ is height above equilibrium, $u(t,x)$ is horizontal water velocity
\item FIXME
\end{itemize}
\end{frame}


\begin{frame}{shallow water: illustration}

\begin{itemize}
\item dam break example

\hfill \includegraphics[width=0.75\textwidth]{figs/leveque13p4}

\end{itemize}
\end{frame}


\begin{frame}[fragile]
\frametitle{xx}

\begin{itemize}
\item xx:

\medskip
\begin{Verbatim}[fontsize=\scriptsize]
>> size(A)
ans =
      201        5
\end{Verbatim}

%\includegraphics[width=0.55\textwidth]{figs/legendre} \quad 
\end{itemize}
\end{frame}




\begin{frame}{summary}

\begin{itemize}
\item since the 1990s there is a recommended finite volume solver for linear and nonlinear hyperbolic systems
\item sometimes called a ``high-resolution Godunov method''
\item it consists of three things:
    \begin{itemize}
    \item[$\circ$] \alert{finite volumes}:
        \begin{itemize}
        \item the conservation law form is spatially integrated
        \item the discrete unknowns are the cell averages
        \item the flux is needed at faces between cells
        \end{itemize}
    \item[$\circ$] \alert{Riemann solvers}:
        \begin{itemize}
        \item the ``Riemann problem'' has different cell values on each side of a face
        \item the Riemann solver is provided by the user
        \item it determines the local wave structure (e.g.~rarefaction and shock waves) and then evaluates the flux on the face
        \item it can be exact or approximate
        \end{itemize}
    \item[$\circ$] \alert{slope limiters}:
        \begin{itemize}
        \item in first-order upwinding the solution is a constant (slope $= 0$); this is ``classical Godunov''
        \item for higher order the solution inside a cell is ``reconstructed'' as a sloped function
        \item \dots but the slope is (nonlinearly) limited so that the solution does not propagate oscillations
        \item ``does not propagate oscillations'' may mean \emph{total variation diminishing} (TVD)
        \end{itemize}
    \end{itemize}
\end{itemize}
\end{frame}

\end{document}

