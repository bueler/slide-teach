% Copyright 2023  Ed Bueler

\documentclass[10pt,hyperref]{beamer}

\mode<presentation>
{
  \usetheme{Madrid}

  \usecolortheme{beaver}

  \setbeamercovered{transparent}
  
  \setbeamerfont{frametitle}{size=\large}
}

\setbeamercolor*{block title}{bg=red!10}
\setbeamercolor*{block body}{bg=red!5}

\usepackage[english]{babel}
\usepackage[latin1]{inputenc}
\usepackage{times}
\usepackage[T1]{fontenc}
% Or whatever. Note that the encoding and the font should match. If T1
% does not look nice, try deleting the line with the fontenc.

\usepackage{empheq,bm,xspace,fancyvrb}
\usepackage{hyperref}

% If you wish to uncover everything in a step-wise fashion, uncomment
% the following command: 
%\beamerdefaultoverlayspecification{<+->}

\newcommand{\bb}{\mathbf{b}}
\newcommand{\bc}{\mathbf{c}}
\newcommand{\bbf}{\mathbf{f}}
\newcommand{\bl}{\bm{\ell}}
\newcommand{\br}{\mathbf{r}}
\newcommand{\bs}{\mathbf{s}}
\newcommand{\bx}{\mathbf{x}}
\newcommand{\by}{\mathbf{y}}
\newcommand{\bv}{\mathbf{v}}
\newcommand{\bu}{\mathbf{u}}
\newcommand{\bw}{\mathbf{w}}

\newcommand{\bzero}{\bm{0}}

\newcommand{\CC}{\mathbb{C}}
\newcommand{\RR}{\mathbb{R}}

\newcommand{\ddt}[1]{\ensuremath{\frac{\partial #1}{\partial t}}}
\newcommand{\ddx}[1]{\ensuremath{\frac{\partial #1}{\partial x}}}
\renewcommand{\t}[1]{\texttt{#1}}
\newcommand{\Matlab}{\textsc{Matlab}\xspace}
\newcommand{\Octave}{\textsc{Octave}\xspace}
%\newcommand{\MO}{\Matlab/\Octave}
\newcommand{\MO}{\Matlab}
\newcommand{\eps}{\epsilon}

\newcommand{\twovect}[4]{\ensuremath{{#1}_{#2} =
                            \begin{bmatrix} #3 \\ #4 \end{bmatrix}}}

\newcommand{\mfile}[1]{
\VerbatimInput[frame=single,label=\fbox{\scriptsize \textsl{\,#1\,}},fontfamily=courier,fontsize=\scriptsize]{#1}
}

\newcommand{\mfiletiny}[1]{
\VerbatimInput[frame=single,label=\fbox{\scriptsize \textsl{\,#1\,}},fontfamily=courier,fontsize=\tiny]{#1}
}


\title{Which linear systems can be solved optimally?}

\author{Ed Bueler}

\institute[]{UAF Math 692 Scalable Seminar}

\date{Spring 2023}


\begin{document}

\begin{frame}
  \maketitle
\end{frame}

\begin{frame}{Outline}
  \tableofcontents[hideallsubsections]
\end{frame}

\section{how fast is the basic $Ax$ matrix-vector product operation?}

\begin{frame}{example linear systems}

\begin{itemize}
\item x
\end{itemize}
\end{frame}

\AtBeginSection[]
{
  \begin{frame}<beamer>
    \frametitle{Outline}
    \tableofcontents[currentsection,hideallsubsections]
  \end{frame}
}

\section{complexity of Gauss elimination for systems}

\begin{frame}{example linear systems}

\begin{itemize}
\item x
\end{itemize}
\end{frame}

\section{tridiagonal and other banded matrices}

\begin{frame}{example linear systems}

\begin{itemize}
\item x
\end{itemize}
\end{frame}

\section{circulant matrices}

\begin{frame}{example linear systems}

\begin{itemize}
\item x
\end{itemize}
\end{frame}

\section{sparse storage?}

\section{paradigm: preconditioned Krylov iterations}

\section{multigrid: a teaser}

\begin{frame}{example linear systems}

\begin{itemize}
\item suppose we want to solve the linear system
\begin{equation}
A \bx = \bb \label{introsystem}
\end{equation}
where $A\in \RR^{m\times m}$ and $\bb\in \RR^m$
\item the goal is to find $\bx\in \RR^m$
\item  throughout these notes we use 2 linear system examples:
  \begin{itemize}
  \item[LS1] 
\begin{equation*}
\begin{bmatrix} 2 & 1 & 0 \\
                0 & 2 & 1 \\
                1 & 0 & 3 \end{bmatrix}
\begin{bmatrix} x_1 \\ x_2 \\ x_3 \end{bmatrix}
=
\begin{bmatrix} 2 \\ 1 \\ 4 \end{bmatrix}
\end{equation*}
  \item[LS2]
\begin{equation*}
\begin{bmatrix} 1 & 2 & 3 & 0 \\
                2 & 1 &-2 &-3 \\
               -1 & 1 & 1 & 0 \\
                0 & 1 & 1 &-1 \end{bmatrix}
\begin{bmatrix} x_1 \\ x_2 \\ x_3 \\ x_4 \end{bmatrix}
=
\begin{bmatrix} 7 \\ 1 \\ 1 \\ 3 \end{bmatrix}
\end{equation*}
  \end{itemize}
\item it is trivial to find solutions of LS1, LS2 using the ``\texttt{x=A$\backslash$b}'' black box in \Matlab (or similar)
\item LS1 and LS2 stand-in for the large linear systems we get from applying finite difference (FD) schemes to ODE and PDE problems
\end{itemize}
\end{frame}

\end{document}

