\documentclass[11pt]{amsart}
%prepared in AMSLaTeX, under LaTeX2e
\addtolength{\oddsidemargin}{-.5in}
\addtolength{\evensidemargin}{-.5in}
\addtolength{\topmargin}{-.2in}
\addtolength{\textwidth}{0.8in}
\addtolength{\textheight}{0.8in}
\newcommand{\normalspacing}{\renewcommand{\baselinestretch}{1.1}
        \tiny\normalsize}

\newtheorem*{thm}{Theorem}
\newtheorem*{defn}{Definition}
\newtheorem*{example}{Example}
\newtheorem*{problem}{Problem}
\newtheorem*{remark}{Remark}

\usepackage{amssymb,verbatim,alltt,xspace}
\newcommand{\mtt}{\texttt}
\newcommand{\mfile}[1]
{\medskip\begin{quote}\scriptsize \begin{alltt}\input{#1.m}\end{alltt} \normalsize\end{quote}\medskip}

\usepackage[final]{graphicx}
\newcommand{\mfigure}[1]{\includegraphics[height=3in,
keepaspectratio=true]{#1.eps}}

\usepackage{hyperref}

% macros
\newcommand{\bb}{\mathbf{b}}
\newcommand{\by}{\mathbf{y}}

\newcommand{\CC}{\mathbb{C}}
\newcommand{\Div}{\nabla\cdot}
\newcommand{\eps}{\epsilon}
\newcommand{\grad}{\nabla}
\newcommand{\ZZ}{\mathbb{Z}}
\newcommand{\ip}[2]{\ensuremath{\left<#1,#2\right>}}
\newcommand{\lam}{\lambda}
\newcommand{\lap}{\triangle}
\newcommand{\tb}{\textsc{Morton \& Mayers}}
\newcommand{\RR}{\mathbb{R}}
\newcommand{\pexer}[2]{\bigskip\noindent\textbf{#1.} Exercise #2 in \tb.}
\newcommand{\prob}[1]{\medskip\noindent\textbf{#1.} }
\newcommand{\epart}[1]{\medskip\noindent\textbf{(#1)} }
\newcommand{\note}[1]{[\scriptsize #1 \normalsize]}

\newcommand{\Matlab}{\textsc{Matlab}\xspace}
\newcommand{\Octave}{\textsc{Octave}\xspace}
\newcommand{\pylab}{\textsc{pylab}\xspace}
\newcommand{\MOP}{\textsc{MOP}\xspace}

\begin{document}
\scriptsize \noindent Math 615 Numerical Analysis of DEs \hfill  Bueler; \today
\normalsize\bigskip

\Large\centerline{\textbf{Assignment \#5}}
\normalsize\bigskip

\centerline{Due \emph{Monday, 10 March, 2014} at start of class.}
\bigskip
\thispagestyle{empty}
\normalspacing

\begin{center}
\emph{See the online slides:} \quad \url{http://bueler.github.io/M615S14/twopoint.pdf}
\end{center}
\bigskip

\prob{1}  Solve by-hand this ODE BVP to find $y(x)$:
	$$y'' + 2 y' + 2 y = 0, \qquad y(0)=1, \quad y(1)=0.$$

\prob{2}  Recall Example 3 in the slides, an impossible-to-solve ODE BVP.  Nonetheless there are some values of $A$ in the following problem which allow a solution:  find $y(x)$ if
  	$$y'' + \pi^2 y = 0, \qquad y(0)=1, \quad y(1)=A.$$
What values of $A$ are allowed?  For an allowed value of $A$, how many solutions are there?

\prob{3} Apply the finite difference method to solve this ODE BVP:
	$$y'' + \sin(5 x) y = x^3 - x, \qquad y(0)=0, \quad y(1) = 0.$$
In particular, use $J=10$, $\Delta x = 1/J$, and $x_j = j \Delta x$ for $j=0,\dots,J$.  Construct the system
	$$A\,\by = \bb$$
where $A$ is a $(J+1)\times(J+1)$ matrix, $\by = \{Y_j\}$ approximates the unknowns $\{y(x_j)\}$, and $\bb$ contains the right-side function ``$x^3-x$'' in the ODE.  Arrange things so that the first equation in the system represents the boundary condition ``$y(0)=0$'' and the last equation the condition ``$y(1)=0$''.  The remaining equations in the system will each hold finite difference approximations of the ODE.  Solve the system to find $\by$, and plot it appropriately.  Write a couple of sentences addressing how to know qualitatively and quantitatively whether your answer is a good approximation.

\prob{4} (\emph{The goal of this problem is to understand shooting.  You will not quite put all parts together, however.  With the knowledge from this problem you could make a program like} \texttt{varheatSHOOT.m}, \emph{which uses bisection to converge to an $A$ value so that $u(1)\approx -2$ to many-digit-accuracy.})

Consider the nonlinear ODE BVP
	$$u'' + u^3 = 0, \qquad u(0) = 1, \quad u(1) = -2.$$
This problem is well-suited to the shooting method.  Specifically, write a \Matlab program that uses an ODE solver to solve the following ODE IVP
	$$u'' + u^3 = 0, \qquad u(0) = 1, \quad u'(0) = A$$
for each of the eleven values $A=-5,-4,\dots,4,5$.  Plot all eleven solutions, and identify on the plot\footnote{Use the \texttt{text} command in \Matlab.} the $A$ value for each curve.  Which two $A$ values make the computed value $u(1)$ bracket the desired boundary condition value ``$u(1)=-2$''?

\end{document}

