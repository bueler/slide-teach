% Copyright 2010, 2014 Ed Bueler

% based on example (directory) by Till Tantau:
%   /usr/share/doc/latex-beamer/examples/a-lecture/

\documentclass[10pt,hyperref]{beamer}

\input{twopoint_style.tex}

\newcommand{\ppt}[1]{\ensuremath{\frac{\partial #1}{\partial t}}}
\newcommand{\ppx}[1]{\ensuremath{\frac{\partial #1}{\partial x}}}
\newcommand{\ppy}[1]{\ensuremath{\frac{\partial #1}{\partial y}}}
\newcommand{\pp}[2]{\ensuremath{\frac{\partial #1}{\partial #2}}}
\newcommand{\ddx}[1]{\ensuremath{\frac{d #1}{dx}}}
\renewcommand{\t}[1]{\texttt{#1}}
\newcommand{\bb}{\mathbf{b}}
\newcommand{\bq}{\mathbf{q}}
\newcommand{\bU}{\mathbf{U}}
\newcommand{\bx}{\mathbf{x}}
\newcommand{\by}{\mathbf{y}}
\newcommand{\bY}{\mathbf{Y}}
\newcommand{\bbf}{\mathbf{f}}
\newcommand{\eps}{\epsilon}
\newcommand{\grad}{\nabla}
\newcommand{\Div}{\nabla\cdot}
\newcommand{\strainrate}{D}
\newcommand{\devstress}{\tau}

\newcommand{\Matlab}{\textsc{Matlab}\xspace}
\newcommand{\Octave}{\textsc{Octave}\xspace}
\newcommand{\pylab}{\textsc{pylab}\xspace}
\newcommand{\MOP}{\textsc{MOP}\xspace}

\newcommand{\exer}[2]{\medskip\noindent \textbf{#1.}\quad #2}

\newcommand{\txtinput}[1]{
\verbatiminput{#1}%a
}

\newcommand{\minput}[1]{
\verbatiminput{#1.m}%a
}

\newcommand{\centerimage}[2]{\begin{center}
\includegraphics[width=#1\textwidth]{#2}
\end{center}}

\newcommand{\minputtiny}[1]{\tiny
\verbatiminput{#1.m}%a
\normalsize}


\lecture[1]{Two-point Boundary Value Problems: Numerical Approaches}{lecture-text}

\date{February 2014}


\begin{document}

\begin{frame}
  \maketitle
\end{frame}


\begin{frame}{abbreviations} 

\begin{itemize}
\item ODE = ordinary differential equation
\item PDE = partial differential equation
\item IVP = initial value problem
\item BVP = boundary value problem
\end{itemize}
\end{frame}


\AtBeginSection[]
{
  \begin{frame}<beamer>
    \frametitle{Outline}
    \tableofcontents[currentsection]
  \end{frame}
}

\section[classical IVPs and BVPs]{classical IVPs and BVPs with by-hand solutions}

\begin{frame}{classical ODE problems: IVP vs BVP}

\noindent\emph{Example 1: ODE IVP}.\qquad find $y(x)$ if
  	$$y'' + 2 y' - 8 y = 0, \qquad y(0)=1, \quad y'(0)=0$$
\bigskip

\noindent\emph{Example 2: ODE BVP}.\qquad find $y(x)$ if
  	$$y'' + 2 y' - 8 y = 0, \qquad y(0)=1, \quad y(1)=0$$

\end{frame}


\begin{frame}{classical ODE problems: IVP vs BVP}

\small
\noindent\emph{Example 1: ODE IVP}.\qquad find $y(x)$ if
  	$$y'' + 2 y' - 8 y = 0, \qquad y(0)=1, \quad y'(0)=0$$

\noindent\emph{Example 2: ODE BVP}.\qquad find $y(x)$ if
  	$$y'' + 2 y' - 8 y = 0, \qquad y(0)=1, \quad y(1)=0$$
\normalsize

  \begin{itemize}
  \item \alert{both problems can be solved by hand}
  \item in fact, the ODE has constant coefficients so we can find \emph{characteristic polynomial} and \emph{general solution} \dots like this:\quad  if $y(x)=e^{rx}$ then $r^2 + 2 r - 8 = (r+4)(r-2) = 0$ so 
  $$y(x) = c_1 e^{-4x} + c_2 e^{2x}$$
  \item \emph{Example 1} gives system $c_1+c_2=1,-4c_1+2c_2=0$ for coefficients; get solution $y(x)=(1/3)e^{-4x}+(2/3)e^{2x}$  % c_2=2/3, c_1=1/3
  \item \emph{Example 2} gives system $c_1+c_2=1,e^{-4} c_1+ e^2 c_2=0$ for coefficients; get solution $y(x)=(1-e^{-6})^{-1} e^{-4 x} + (1-e^6)^{-1} e^{2x}$
  \end{itemize}
\end{frame}


\begin{frame}[fragile]{viewing solns with \Matlab} 
\small
\begin{verbatim}
x = 0:.001:1;
y1 = exp(-4*x);  y2 = exp(2*x);
yIVP = (1/3)*y1 + (2/3)*y2;
yBVP = (1/(1-exp(-6)))*y1 + (1/(1-exp(6)))*y2;
plot(x,yIVP,x,yBVP),  grid on
legend('IVP soln','BVP soln')
\end{verbatim}

\centerimage{0.6}{ivpbvp-crop}
\end{frame}


\begin{frame}{obvious name: ``two-point BVP''} 


\normalsize

\begin{itemize}
\item \emph{Example 2} above is called a ``two-point BVP''
\item a two-point BVP includes an ODE and the value(s) of the solution at two different locations
\item the ODE can be of any order, as long as it is at least \emph{two}, because first-order ODEs cannot satisfy two conditions (generally)
\item \emph{but} there is no guarantee that a two-point BVP can be solved (see below)
\item we will also consider boundary value problems for PDEs in this course (i.e.~problems including no initial values)
\end{itemize}
\end{frame}


\begin{frame}{a standard manipulation of a 2nd order ODE} 

Consider the general linear 2nd-order ODE:
\begin{equation}\label{genlinearsecond}
y'' + p(x) y' + q(x) y = r(x)
\end{equation}

Also consider the general 2nd-order ODE:
\begin{equation}\label{gensecond}
y'' = f(x,y,y')
\end{equation}

\begin{itemize}
\item \alert{these can be written as systems of coupled 1st-order ODEs}
\item equation \eqref{genlinearsecond} is equivalent to
	$$\begin{pmatrix}
	y' \\ v'
	\end{pmatrix} = \begin{pmatrix}
	v \\ - p(x) v - q(x) y + r(x)
	\end{pmatrix}$$
\item equation \eqref{gensecond} is equivalent to
	$$\begin{pmatrix}
	y' \\ v'
	\end{pmatrix} = \begin{pmatrix}
	v \\ f(x,y,v)
	\end{pmatrix}$$
\item first order systems are the form in which to apply a numerical ODE solver
\end{itemize}
\end{frame}


\begin{frame}{why IVP are \emph{better} problems than BVPs} 

\begin{itemize}
\item \alert{IVPs have unique solutions}
\item we say they are ``well-posed''; specifically:

\begin{theorem}
 Consider the system of ODEs
\begin{equation}\label{genode}
\frac{d\by}{dt} = \bbf(t,\by),
\end{equation}
where $\by(t)=(y_1(t),\dots,y_d(t))$ and $\bbf=(f_1,\dots,f_d)$ are vector-valued functions.  If $\bbf$ is continuous for $t$ in an interval around $t_0$ and for $\by$ in some region around $\by_0$, and if $\partial f_i/\partial y_j$ is continuous for the same inputs and for all $i$ and $j$, then the IVP consisting of \eqref{genode} and $\by(t_0)=\by_0$ has a unique solution $\by(t)$ for at least some small interval $t_0-\eps < t < t_0+\eps$ for some $\eps>0$.
\end{theorem}

\item given comments on last slide, this theorem also covers IVPs for 2nd-order scalar ODEs
\end{itemize}
\end{frame}

\begin{frame}{warning about apparently-easy BVPs} 

\noindent\emph{Example 3: ODE BVP}.\qquad find $y(x)$ if
  	$$y'' + \pi^2 y = 0, \qquad y(0)=1, \quad y(1)=0$$

\begin{itemize}
\item this turns out to be \alert{impossible} \dots there is no such $y(x)$
\item in fact, the general solution to the ODE is
	$$y(x) = c_1 \cos(\pi x) + c_2 \sin(\pi x)$$
so the first boundary condition implies $c_1=1$
\item \dots but then the second condition says
	$$\text{``}\qquad 0 = y(1) = -1 + c_2 \sin(\pi) \qquad \text{''}$$
and this has no solution because $\sin(\pi)=0$
\item this is a constant-coefficient problem for which all the ``parts'' are ``well-behaved'' \dots but it is a BVP
\end{itemize}
\end{frame}


\section[serious problem]{a serious problem: a BVP for equilibrium heat}


\begin{frame}{an equilibrium heat example} 

\begin{itemize}
\item as noted in lecture and by Morton \& Mayers, a PDE like this is a general description of heat flow in a rod:
\begin{equation}\label{pde}
\rho c \ppt{u} = \ppx{}\left(k(x) \ppx{u}\right) + r(x) u + s(x)
\end{equation}
\item recall that, roughly speaking, $\rho$ is a density, $c$ a specific heat, $k(x)$ a conductivity, $r(x)$ a reaction coefficient, and $s(x)$ is an external source of heat
\end{itemize}
\end{frame}

\begin{frame}{an equilibrium heat example, cont} 

\begin{itemize}
\item \emph{equilibrium} means no change in time; the equilibrium version of \eqref{pde} is this:
\begin{equation*}
0 = \ppx{}\left(k(x) \ppx{u}\right) + r(x) u + s(x)
\end{equation*}
\item we can use ordinary derivative notation; the equilibrium equation is an ODE:
\begin{equation}\label{seriousode}
\left(k(x) u'\right)' + r(x) u = - s(x)
\end{equation}
\item suppose the rod has length $L$
\item example boundary values are \emph{(i)} insulation at the left end and \emph{(ii)} zero temperature at the right end:
\begin{equation}\label{seriousbcs}
u'(0)=0, \qquad u(L)=0
\end{equation}
\end{itemize}
\end{frame}

\begin{frame}{an equilibrium heat example, cont} 

\begin{itemize}
\item some concrete choices in my example include $L=3$ and:
\begin{gather*}
k(x) = \frac{1}{2} \arctan(20 (x-1)) + 1, \\
r(x) = r_0 = \frac{1}{2}, \qquad s(x) = e^{-(x-2)^2}
\end{gather*}
\end{itemize}

\centerimage{0.7}{nonconstant-crop90}
\end{frame}


\begin{frame}[fragile]{an equilibrium heat example, cont}

\begin{itemize}
\item code used to produce the previous picture
\end{itemize}
\small
\begin{verbatim}
L  = 3;
k  = @(x) 0.5 * atan((x-1.0) * 20.0) + 1.0;
r0 = 0.5;
s  = @(x) exp(-(x-2.0).^2);

J = 300;
dx = L / J;
x = 0:dx:L;
plot(x,k(x),x,r0*ones(size(x)),x,s(x))
grid on,  xlabel x
legend('k(x)','r(x)=r_0','s(x)')
\end{verbatim}
\end{frame}


\begin{frame}{an equilibrium heat example, cont} 

\begin{itemize}
\item we have set up a non-constant-coefficient boundary value problem to solve:
\begin{equation}\label{serious}
\left(k(x) u'\right)' + r_0 u = - s(x), \qquad u'(0)=0, \quad u(3) = 0
\end{equation}
\item $u(x)$ represents the equilibrium distribution of temperature in a rod with these properties:
  \begin{itemize}
  \item conductivity $k(x)$: the first third $[0,1]$ is a material with much lower conductivity than the last two-thirds $[2,3]$
  \item reaction rate $r_0>0$: constant rate of linear-in-temperature heating
  \item source term $s(x)$: an external heat source concentrated around $x=2$
  \end{itemize}
\item \emph{Question}: what is $u(0)$, the temperature at the left end?
\item I will call this my ``serious problem'', and solve it numerically two different ways
\end{itemize}
\end{frame}


\begin{frame}{plan from here} 

\begin{enumerate}
\item introduce finite difference approach on really-easy ``toy'' two-point BVP
\item introduce shooting method on same toy problem
\item demonstrate both approaches on ``serious problem''
\end{enumerate}
\end{frame}


\section[finite difference]{finite difference solution of two-point BVPs}

\begin{frame}{finite differences} 

\begin{itemize}
\item finite difference methods for two-point BVPs generalize to PDEs \dots as demonstrated in the rest of Math 615
\item here we are just solving ODEs
\bigskip\bigskip

\item recall:
	$$\frac{f(x-h) - 2 f(x) + f(x+h)}{h^2} = f''(x) + \frac{f^{(4)}(\nu)}{12} h^2$$
\end{itemize}
\end{frame}


\begin{frame}{toy example problem} 

\begin{itemize}
\item consider this easy BVP:
    $$y'' = 12 x^2, \qquad y(0)=0, \quad y(1) = 0$$
\item it has exact solution $y(x)=x^4-x$
\item \alert{please check my last claim}
\item \alert{make sure you could solve this yourself!}
\end{itemize}
\end{frame}


\begin{frame}{toy example: approximated by finite differences} 

\begin{itemize}
\item cut up the interval $[0,1]$ into $J$ subintervals:
	$$\Delta x = 1/J$$
	$$x_j = 0 + (j-1) \Delta x \qquad\quad(j=1,\dots,J+1)$$
\item note that my indices run from $j=1$ to $j=J+1$
\item let $Y_j$ be the approximation to $y(x_j)$
\item for each of $j=2,\dots,J$ we approximate
    $$y'' = 12 x^2$$
by
	$$\frac{Y_{j-1} - 2 Y_j + Y_{j+1}}{\Delta x^2} = 12 x_j^2$$
\item the boundary conditions are: $Y_1 = 0$, $Y_{J+1} = 0$
\end{itemize}
\end{frame}


\begin{frame}{toy example: approximated by finite differences, cont} 

\begin{itemize}
\item so now we have a linear system of $J+1$ equations in $J+1$ unknowns:
\begin{align*}
Y_1 & = 0 \\
Y_1 - 2 Y_2 + Y_3 &= 12 x_2^2 \Delta x^2 \\
Y_2 - 2 Y_3 + Y_4 &= 12 x_3^2 \Delta x^2 \\
\vdots &\qquad \vdots \\
Y_{J-1} - 2 Y_J + Y_{J+1} &= 12 x_J^2 \Delta x^2 \\
Y_{J+1} &= 0
\end{align*}
\end{itemize}
\end{frame}


\begin{frame}{toy example: as matrix problem} 

\begin{itemize}
\item this is a matrix problem:
   $$\begin{bmatrix}
   1  &  0 &  0 & 0 & \dots & 0 \\
   1  & -2 &  1 & 0 & \dots & 0 \\
   0  &  1 & -2 & 1 &       & 0 \\
   \vdots &&    & \ddots &   &  \\
      &    &    &  1 &    -2 & 1 \\
   0  & \dots & &  0 &     0 & 1
   \end{bmatrix}
   \begin{bmatrix}
   Y_1 \\ Y_2 \\ Y_3 \\ \vdots \\ Y_J \\ Y_{J+1}
   \end{bmatrix}
   =
   \begin{bmatrix}
   0 \\ 12 x_2^2 \Delta x^2 \\ 12 x_3^2 \Delta x^2 \\ \vdots \\ 12 x_J^2 \Delta x^2 \\ 0
   \end{bmatrix} $$
\bigskip


\item i.e.
	$$A\,\bY = \bb$$
\end{itemize}
\end{frame}


\begin{frame}{toy example: as matrix problem in \Octave} 

\begin{itemize}
\item the matrix $A$ is \emph{tridiagonal}
\item which is usually true of finite difference methods for two-point boundary value problems for second order ODEs
\item $A$ has lots of zero entries
\item use \Matlab's \texttt{sparse} to store it
\item the \emph{locations} of nonzero entries, and the nonzero values, are stored; this saves space
\item the backslash command in \Matlab is an ``expert system''
  \begin{itemize}
  \item[$\circ$] recognizes sparsity pattern
  \item[$\circ$] exploits it to speed up matrix/vector operations
  \end{itemize}
\item use \texttt{spy} and \texttt{full} to see sparse matrices
\end{itemize}
\end{frame}


\begin{frame}[fragile]{toy example: as matrix problem in \Octave, cont} 

\begin{itemize}
\item setting up the matrix problem looks like:
\small
\begin{verbatim}
J = 10;  dx = 1/J;  x = (0:dx:1)';
b = zeros(J+1,1);
b(2:J) = 12 * dx^2 * x(2:J).^2;
A = sparse(J+1,J+1);
A(1,1) = 1.0;  A(J+1,J+1) = 1.0;
for j=2:J
  A(j,[j-1, j, j+1]) = [1, -2, 1];
end
\end{verbatim}
\normalsize
\item solving the matrix problem looks like:
\small
\begin{verbatim}
Y = A \ b;   % solve A Y = b
\end{verbatim}
\normalsize
\item plot on next page from 
\small
\begin{verbatim}
% also get exact soln on fine grid:
xf = 0:1/1000:1;  yexact = xf.^4 - xf;
plot(x,Y,'o','markersize',12,xf,yexact)
grid on, xlabel x, legend('finite diff','exact')
\end{verbatim}
\end{itemize}
\end{frame}


\begin{frame}{toy example: as matrix problem in \Octave, cont, cont} 

\begin{itemize}
\item gives result which is better than we have any reason to expect:
\end{itemize}

\centerimage{0.9}{toyfd-crop90}

\end{frame}



\begin{frame}{toy example with finite differences: brief analysis} 

\emph{regarding the result on the previous slide}:
\begin{itemize}
\item recall the \emph{exact} solution is $y(x) = x^4 - x$
\item and \small
	$$\frac{f(x-h) - 2 f(x) + f(x+h)}{h^2} = f''(x) + \frac{f^{(4)}(\nu)}{12} h^2$$
\normalsize
\item applied to $f(x)=y(x)$, for which $y^{(4)}(x) = 24$, we see that the finite difference approximation to the second derivative in the ODE $y'' = 12 x^2$ has error at most 
	$$\frac{y^{(4)}(\nu)}{12} \Delta x^2 = \frac{24}{12} (0.1)^2 = 0.02$$
because $\Delta x = 0.1$
\item this is a rare case where the truncation error is known!
\end{itemize}
\end{frame}


\begin{frame}{toy example with finite differences: brief analysis, cont} 

\begin{itemize}
\item let $e_j = Y_j - y(x_j)$, the \emph{error} we care about
\item by subtraction,
	$$\frac{e_{j-1} - 2 e_j + e_{j+1}}{\Delta x^2} = 0.02$$
and $e_0 = e_{J+1} = 0$
\item so (after bit of not-too-hard thought)
	$$e_j = 0.01 x_j (x_j - 1)$$
\item so
	$$\max_j |Y_j - y(x_j)| = \max_j |e_j| = 0.0025$$
\item which explains why the picture a few slides back was good \dots but showed slight errors close to screen resolution
\end{itemize}
\end{frame}


\section[shooting]{shooting to solve two-point BVPs}

\begin{frame}{toy example problem again: shooting} 

\begin{itemize}
\item recall this ``toy'' ODE BVP:
    $$y'' = 12 x^2, \qquad y(0)=0, \quad y(1) = 0$$
which has exact solution $y(x)=x^4-x$
\item this time we think: \emph{if only it were an ODE IVP then we could apply a numerical ODE solver like} \Matlab's \texttt{ode45}
\item indeed, this ODE IVP
    $$w'' = 12 x^2, \qquad w(0)=0, \quad w'(0) = A$$
\emph{can} be solved by a numerical ODE solver, for any $A$
\item solving this ODE IVP involves \alert{``aiming''} by guessing an initial slope $w'(0)=A$
\item \alert{``hitting the target''} is getting the desired boundary value $w(1)=0$
\item ``aiming'' and ``hitting the target'' is \emph{shooting}
\end{itemize}
\end{frame}


\begin{frame}{toy example shooting, cont} 

\begin{itemize}
\item for illustrating the method on this easy problem, I'll skip using a numerical ODE solver because the ODE IVP
    $$w'' = 12 x^2, \qquad w(0)=0, \quad w'(0)=A$$
has a solution we can get by-hand:
    $$w(x) = x^4 + A x$$
\item plotting for $A=-2.5,-1.5,-0.5,0.5,1.5$ gives this figure:
\end{itemize}

\centerimage{0.65}{toyshoot-crop90}
\end{frame}


\begin{frame}{toy example shooting, cont, cont} 

\begin{itemize}
\item we have ``aimed'' (by choosing $A$) and ``shot'' five times
\item a ``shot'' is a computation of the solution to an ODE IVP
  \begin{itemize}
  \item[$\circ$] generally this would be a numerical solution
  \end{itemize}
\item on previous slide we missed every time
\item but we have bracketed the correct right-hand boundary condition $y(1)=0$ with the two values $A=-1.5$ and $A=-0.5$
\item a numerical \emph{equation} solver can refine the search to converge to the correct $A$ value
\end{itemize}
\end{frame}


\begin{frame}{shooting: solving the boundary condition equation} 

\begin{itemize}
\item recall our ODE BVP
    $$y'' = 12 x^2, \qquad y(0)=0, \quad y(1)=0$$
is replaced by this ODE IVP when ``shooting'':
\begin{equation}\label{shootA}
w'' = 12 x^2, \qquad w(0)=0, \quad w'(0)=A
\end{equation}
\item the $x=1$ endpoint value of $w(x)$ \alert{is a function of $A$}:
	$$F(A) = \Big(w(1), \text{ where } w \text{ solves \eqref{shootA}}\Big)$$
\item and so we solve this equation because we want $y(1)=0$:
	$$F(A) = 0$$
\item in this easy problem, $w(x)=x^4+Ax$
\item so we solve $F(A) = 1 + A = 0$ and get $A=-1$ 
\item generally we solve $F(A)=0$ numerically, e.g.~by the \emph{bisection} or \emph{secant} methods 
\end{itemize}
\end{frame}


\begin{frame}{shooting: general strategy for two-point ODE BVPs} 

\begin{itemize}
\item identify one end of the interval $x=b$ as the target
\item at the other end $x=a$, identify some additional initial conditions which would give a well-posed ODE IVP
\item for various guesses of those additional initial conditions, ``shoot'' by solving the corresponding ODE IVP from $x=a$ to $x=b$
\item ask whether you ``hit the target'' by asking whether the boundary conditions at $x=b$ are satisfied
\item automate the adjustment process by using an equation solver (e.g.~bisection or secant method) on the equation that says ``the discrepancy between the solution of the ODE IVP at $x=b$ and the desired boundary conditions at $x=b$, as a function of the additional initial condition $A$, should be zero: $F(A)=0$''
\end{itemize}
\end{frame}


\section[serious example: solved]{a more serious example: solutions}

\begin{frame}{recall the serious example}

\begin{itemize}
\item recall the ``serious'' non-constant-coefficient BVP:
\begin{equation}\label{seriousagain}
\left(k(x) u'\right)' + r_0 u = - s(x), \qquad u'(0)=0, \quad u(3) = 0,
\end{equation}
\item $u(x)$ is the equilibrium temperature in a rod
\item the conductivity $k(x)$ has a big jump at $x=1$ and the heat source $s(x)$ is concentrated near $x=2$:
\end{itemize}

\centerimage{0.6}{nonconstant-crop90}

\end{frame}


\begin{frame}{finite differences: need staggered grid} 

\begin{itemize}
\item finite difference approach first
\item as before: $J$ subintervals, $\Delta x = 1/J$, and
	$$x_j = (j-1) \Delta x \qquad \text{ for } j=1,\dots,J+1$$
\item let $U_j$ be our finite diff.~approx.~to $u(x_j)$
\item let $k_j = k(x_j)$ and $s_j = s(x_j)$; we know these exactly
\item note: if $q(x) = -k(x) u'(x)$, i.e.~Fourier's law for heat flow, then we are solving 
	$$- q' + r_0 u = -s(x)$$
\item the finite difference version looks like
	$$- \frac{q_{j+1/2} - q_{j-1/2}}{\Delta x} + r_0 U_j = - s(x_j)$$
\item or
	$$\frac{k(x_{j+1/2}) \frac{U_{j+1} - U_j}{\Delta x} - k(x_{j-1/2}) \frac{U_{j} - U_{j-1}}{\Delta x}}{\Delta x} + r_0 U_j = - s(x_j)$$
\end{itemize}
\end{frame}


\newcommand{\shalf}{\frac{1}{2}}

\begin{frame}{finite differences: need staggered grid, cont} 

\begin{itemize}
\item or
	$$\frac{k_{j+\shalf} (U_{j+1} - U_j) - k_{j-\shalf} (U_{j} - U_{j-1})}{\Delta x^2} + r_0 U_j = - s_j$$
\item or (clear denominators)
	$$k_{j+\shalf} (U_{j+1} - U_j) - k_{j-\shalf} (U_{j} - U_{j-1}) + r_0 \Delta x^2 U_j = - s_j \Delta x^2$$
\item or
	$$k_{j-\shalf} U_{j-1} - \left(k_{j-\shalf}+k_{j+\shalf} - r_0 \Delta x^2\right) U_j + k_{j+\shalf} U_{j+1} = - s_j \Delta x^2$$
\item like the ``toy'' example earlier, this last form is a tridiagonal matrix equation $A\bU = \bb$
\item note we evaluate the conductivity $k(x)$, and the flux $q$, on the staggered grid (i.e.~$x_{j+\shalf}$ and $x_{j-\shalf}$)
\item the deeper reason \emph{why} we use the staggered grid will be revealed later in class \dots
\end{itemize}
\end{frame}


\begin{frame}{finite differences: remember the boundary conditions} 

\begin{itemize}
\item recall we have boundary condition $u'(0)=0$
\item approximate this by 
	$$\frac{U_2 - U_1}{\Delta x} = 0$$
\item or 
	$$- U_1 + U_2 = 0$$
\item we will see there is a more-accurate way later \dots
\item also we have $u(L) = 0$ so
	$$U_{J+1} = 0$$
\end{itemize}
\end{frame}


\begin{frame}[fragile]{finite differences for the ``serious problem''} 

\begin{itemize}
\item now for an actual code: see \texttt{varheatFD.m} online
\item the ODE setup:
\scriptsize
\begin{verbatim}
L  = 3;
k  = @(x) 0.5 * atan((x-1.0) * 20.0) + 1.0;
s  = @(x) exp(-(x-2.0).^2);
r0 = 0.5;

dx    = L / J;
x     = (0:dx:L)';             % regular grid
xstag = ((dx/2):dx:L-(dx/2))'; % staggered grid
kstag = k(xstag);              % k(x) on staggered grid
\end{verbatim}
\normalsize
\item the matrix problem setup:
\scriptsize
\begin{verbatim}
% right side is J+1 length column vector
b = [0; - dx^2 * s(x(2:J)); 0];

% matrix is tridiagonal
A = sparse(J+1,J+1);
A(1,[1 2]) = [-1.0 1.0];
for j=1:J-1
  A(j+1,j)   = kstag(j);
  A(j+1,j+1) = - kstag(j) - kstag(j+1) + r0 * dx^2; 
  A(j+1,j+2) = kstag(j+1);
end
A(J+1,J+1) = 1.0;
\end{verbatim}
\normalsize
\end{itemize}
\end{frame}


\begin{frame}{finite differences for the ``serious problem'', cont} 

\begin{itemize}
\item it is good to use ``\texttt{spy(A)}'' at this point to see the matrix structure; this is the $J=10$ case
\end{itemize}

\centerimage{0.55}{spyFD-crop}
\end{frame}


\begin{frame}[fragile]{finite differences for the ``serious problem'', cont, cont} 

\begin{itemize}
\item the matrix solve:
\small
\begin{verbatim}
U = A \ b;       % soln is J+1 column vector
\end{verbatim}
\normalsize
\bigskip

\item the plot details:
\small
\begin{verbatim}
figure(1)
plot(x,k(x),'r',x,s(x),'b',...
     x,U','g*','markersize',3)
grid on,  xlabel x
legend('k(x)','s(x)','solution U_j')
\end{verbatim}
\normalsize
\end{itemize}
\end{frame}

\begin{frame}{finite difference solution to ``serious problem''} 

\begin{itemize}
\item the picture when $J=60$:
\end{itemize}

\centerimage{0.7}{resultFD-crop}
\end{frame}

\begin{frame}{finite difference solution to ``serious problem'', cont} 

\begin{itemize}
\item recall our concrete goal was to estimate $u(0)$ 
\item clearly we should try different $J$ values to estimate:
\medskip
\begin{center}
\begin{tabular}{l|r}
J & estimate of $u(0)$ \\ \hline
10 & -13.86507 \\
20 & -7.20263 \\
60 &  -5.66666 \\
200 & -5.27443 \\
1000 & -5.15199 \\
4000 & -5.12965
\end{tabular}
\end{center}
\bigskip

\item this suggests that $u(0)\approx -5.13$
\item \emph{How do we know how wrong we are?}
\end{itemize}

\end{frame}

\begin{frame}[fragile]{shooting for the ``serious problem''} 

\begin{itemize}
\item shooting is implemented in this code online:
  \begin{itemize}
  \item \texttt{varheatSHOOT.m}
  \end{itemize}
\bigskip

\item the setup:
\scriptsize
\begin{verbatim}
L = 3;
k = @(x) 0.5 * atan((x-1.0) * 20.0) + 1.0;
s = @(x) exp(-(x-2.0).^2);
r0 = 0.5;

% ODE  Y' = G(x,Y)  is described by this right-hand side
G = @(x,Y) [- Y(2) / k(x);       % Y(1) = u 
            r0 * Y(1) + s(x)];   % Y(2) = q

% bracket unknown u(0)
a = -10.0;  % produces u(3) which is too high
b =   0.0;  %      ... u(3) which is too low
\end{verbatim}
\normalsize
\end{itemize}
\end{frame}

\begin{frame}[fragile]{shooting for the ``serious problem'', cont} 

\begin{itemize}
\item the \emph{bisection} implementation, which starts from initial bracket $[a,b]=[-10.0,0.0]$:
\medskip

\small
\begin{verbatim}
N = 100;
for n = 1:N
  c = (a+b)/2;
  [xout,Y] = ode45(G,[0.0 3.0],[c; 0.0]);
  F = Y(end,1);
  if abs(F) < 1e-12 
    break  % we are done
  elseif F >= 0.0 
    a = c;
  else
    b = c;
  end
end
\end{verbatim}
\normalsize
\end{itemize}
\end{frame}


\begin{frame}{shooting solution to ``serious problem''} 

\begin{itemize}
\item the picture:

\centerimage{0.7}{resultSHOOT-crop}
\item default use of \texttt{ode45} gives estimate $u(0)=-5.122257$
\item  \emph{How do we know how wrong we are?}
\end{itemize}
\end{frame}


\begin{frame}{minimal conclusion} 

\begin{itemize}
\item finite difference and shooting methods give comparable solutions to this  ``serious problem''
\item closer inspection of the programs above will help understand the methods
\item better understanding will also follow from doing the exercises on \alert{Assignment \# 5}
\end{itemize}
\end{frame}


\end{document}
